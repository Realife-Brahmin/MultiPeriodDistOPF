\documentclass[11pt]{article}
\usepackage[margin=1in]{geometry}
\usepackage{amsmath}
\usepackage{amssymb}
\usepackage{mathtools}
\usepackage{xcolor}
\usepackage{booktabs}
\usepackage{array}
\usepackage{algorithm}
\usepackage{algpseudocode}

% Define colors for variables
\definecolor{bluevar}{RGB}{0,100,200}     % Blue for current iteration
\definecolor{redvar}{RGB}{200,0,0}        % Red for previous iteration
\definecolor{greenvar}{RGB}{0,128,0}      % Green for dual variables

\newcommand{\blueB}[1]{\textcolor{bluevar}{\mathbf{#1}}}      % Current iteration B
\newcommand{\redB}[1]{\textcolor{redvar}{\mathbf{#1}}}        % Previous iteration B
\newcommand{\greenmu}[1]{\textcolor{greenvar}{\mathbf{#1}}}   % Dual variables \mu

\title{Distributed Dynamic Programming (DDP):\\Forward Pass Decomposition for Copper Plate MPOPF}
\author{}
\date{\today}

\begin{document}
\maketitle

\section{Problem Overview}

Distributed Dynamic Programming (DDP) decomposes a multi-period optimal power flow (MPOPF) problem into $T$ sequential subproblems using a forward pass approach. Each time step $t$ solves a local optimization problem that couples with adjacent time steps through:
\begin{itemize}
    \item \textbf{Backward coupling}: State trajectory from previous time step $t-1$ (current iteration)
    \item \textbf{Forward coupling}: Shadow prices (dual variables) from next time step $t+1$ (previous iteration)
\end{itemize}

The copper plate formulation ignores network constraints and focuses on energy arbitrage with battery storage.

\section{Sets, Indices, and Parameters}

\subsection{Sets and Indices}
\begin{itemize}
    \item $\mathcal{T} = \{1, 2, \ldots, T\}$: Set of time periods
    \item $t \in \mathcal{T}$: Time period index
    \item $k$: DDP iteration index
\end{itemize}

\subsection{System Parameters}
\begin{align}
P_{\text{BASE}} &= 1000 \text{ kW (power base)} \\
E_{\text{BASE}} &= 1000 \text{ kWh (energy base)} \\
\Delta t &= 1.0 \text{ hour (time step duration)}
\end{align}

\subsection{Load and Price Profiles}
\begin{itemize}
    \item $P_L[t]$: Load demand at time $t$ [pu on $P_{\text{BASE}}$]
    \item $C[t]$: Energy price at time $t$ [\$/kWh]
    \item $C_B$: Battery quadratic cost coefficient [\$/kW$^2$/h], typically $10^{-6} \times \min(C[t])$
\end{itemize}

\subsection{Battery Parameters}
\begin{align}
E_{\text{Rated}} &: \text{Battery energy capacity [pu on } E_{\text{BASE}}\text{]} \\
P_{B,R} &: \text{Battery power rating [pu on } P_{\text{BASE}}\text{]} \\
\text{SOC}_{\min}, \text{SOC}_{\max} &: \text{SOC limits (fraction of } E_{\text{Rated}}\text{)} \\
B_0 &: \text{Initial SOC [pu on } E_{\text{BASE}}\text{]} \\
B_{T,\text{target}} &: \text{Terminal SOC constraint (optional) [pu on } E_{\text{BASE}}\text{]}
\end{align}

SOC bounds:
\begin{align}
\underline{B} &= \text{SOC}_{\min} \cdot E_{\text{Rated}} \\
\overline{B} &= \text{SOC}_{\max} \cdot E_{\text{Rated}}
\end{align}

\section{Variable Color Coding}

Following DDP forward pass decomposition:
\begin{itemize}
    \item $\blueB{B[t]}$: Battery SOC at time $t$ in \textbf{current iteration $k$} (blue)
    \item $\redB{B[t]}$: Battery SOC at time $t$ from \textbf{previous iteration $k-1$} (red, used for coupling)
    \item $\greenmu{\mu[t]}$: Shadow price (dual variable) for battery dynamics at time $t$ (green)
    \item $P_{\text{subs}}[t]$: Substation power at time $t$ [pu]
    \item $P_B[t]$: Battery power at time $t$ [pu] (positive = discharge, negative = charge)
\end{itemize}

\section{DDP Forward Pass Subproblem}

\subsection{Subproblem for Time Step $t$ (Iteration $k$)}

Each time step $t \in \{1, \ldots, T\}$ solves the following optimization problem:

\begin{align}
\min_{\substack{P_{\text{subs}}[t], P_B[t], \blueB{B[t]}, \\ P_B[t+1] \text{ (if } t < T\text{)}}} \quad & \underbrace{C[t] \cdot P_{\text{subs}}[t] \cdot \Delta t}_{\text{Energy cost at } t} + \underbrace{C_B \cdot (P_B[t])^2 \cdot \Delta t}_{\text{Battery cost at } t} \notag \\
& + \underbrace{\greenmu{\mu^{k-1}[t+1]} \cdot \left( \redB{B^{k-1}[t+1]} - \blueB{B[t]} + \Delta t \cdot P_B[t+1] \right)}_{\text{Soft coupling with next time step (if } t < T\text{)}}
\end{align}

\textbf{Subject to:}

\textbf{Power Balance:}
\begin{align}
P_{\text{subs}}[t] + P_B[t] &= P_L[t]
\end{align}

\textbf{Battery Dynamics:}
\begin{align}
\blueB{B[t]} - B_{\text{prev}}[t] + \Delta t \cdot P_B[t] &= 0
\end{align}
where:
\begin{align}
B_{\text{prev}}[t] = \begin{cases}
B_0 & \text{if } t = 1 \\
\blueB{B^k[t-1]} & \text{if } t > 1 \quad \text{(from current forward pass)}
\end{cases}
\end{align}

\textbf{SOC Box Constraints (Negative Inequality Form):}
\begin{align}
\underline{B} - \blueB{B[t]} &\leq 0 \quad \text{(i.e., } \blueB{B[t]} \geq \underline{B}\text{)} \\
\blueB{B[t]} - \overline{B} &\leq 0 \quad \text{(i.e., } \blueB{B[t]} \leq \overline{B}\text{)}
\end{align}

\textbf{Battery Power Limits:}
\begin{align}
-P_{B,R} \leq P_B[t] &\leq P_{B,R} \\
-P_{B,R} \leq P_B[t+1] &\leq P_{B,R} \quad \text{(if } t < T\text{)}
\end{align}

\textbf{Terminal SOC Constraint (if $t = T$ and terminal constraint exists):}
\begin{align}
\blueB{B[T]} &= B_{T,\text{target}}
\end{align}

\subsection{Key Formulation Insights}

\subsubsection{Coupling Structure}
\begin{itemize}
    \item \textbf{Backward (state) coupling}: Each subproblem uses $\blueB{B[t-1]}$ from the \textit{current} forward pass (iteration $k$), ensuring consistency within the same pass.
    
    \item \textbf{Forward (cost-to-go) coupling}: The coupling term with time $t+1$ uses:
    \begin{itemize}
        \item $\redB{B^{k-1}[t+1]}$: State from \textit{previous} iteration (parameter)
        \item $\greenmu{\mu^{k-1}[t+1]}$: Shadow price from \textit{previous} iteration (parameter)
        \item $P_B[t+1]$: Battery power at $t+1$ as a \textit{decision variable}
    \end{itemize}
    
    \item The coupling term $\greenmu{\mu[t+1]} \cdot (\redB{B[t+1]} - \blueB{B[t]} + \Delta t \cdot P_B[t+1])$ penalizes violations of the battery dynamics constraint between $t$ and $t+1$.
    
    \item By optimizing $P_B[t+1]$ as a decision variable, each subproblem can "soft-enforce" consistency with the next time step, even though $\redB{B[t+1]}$ is fixed from the previous iteration.
\end{itemize}

\subsubsection{Shadow Price Interpretation}
The dual variable $\greenmu{\mu[t]}$ from the battery dynamics constraint represents:
\begin{itemize}
    \item Marginal value of increasing $\blueB{B[t]}$ by one unit
    \item Shadow price / cost-to-go from state $\blueB{B[t]}$ to terminal time $T$
    \item How much the objective would improve if we had one more unit of energy stored at time $t$
\end{itemize}

For terminal time step $t = T$ with terminal constraint $\blueB{B[T]} = B_{T,\text{target}}$:
\begin{align}
\greenmu{\mu_{\text{total}}[T]} = \greenmu{\mu_{\text{dynamics}}[T]} + \greenmu{\mu_{\text{terminal}}[T]}
\end{align}
where $\greenmu{\mu_{\text{terminal}}[T]}$ is the dual of the terminal constraint.

\subsubsection{No Explicit Dual Update}
Unlike ADMM-based methods, DDP does \textit{not} require a separate dual update step. The shadow prices $\greenmu{\mu[t]}$ are automatically extracted from each subproblem's battery dynamics constraint dual:
\begin{align}
\greenmu{\mu[t]} = -\lambda_{\text{battery\_dynamics}}[t]
\end{align}
where $\lambda_{\text{battery\_dynamics}}[t]$ is the JuMP dual variable (negated for proper economic sign convention).

\section{DDP Algorithm}

\begin{algorithm}[H]
\caption{DDP Forward Pass for Copper Plate MPOPF}
\begin{algorithmic}[1]
\State \textbf{Input:} Problem data $\mathcal{D} = \{T, C[t], P_L[t], B_0, E_{\text{Rated}}, P_{B,R}, \text{SOC}_{\min}, \text{SOC}_{\max}, \Delta t\}$
\State \textbf{Parameters:} $\text{max\_iter}$, $\epsilon_{\text{tol}}$, optional $\greenmu{\mu^0}$ (initial dual variables)
\State
\State \textbf{Initialize} ($k = 0$):
\State \hspace{1em} $\blueB{B^0[t]} \gets B_0$ for all $t \in \mathcal{T}$
\State \hspace{1em} $\greenmu{\mu^0[t]} \gets 0$ (or provided warm start) for all $t \in \mathcal{T}$
\State \hspace{1em} $P_B^0[t] \gets 0$, $P_{\text{subs}}^0[t] \gets 0$ for all $t \in \mathcal{T}$
\State
\For{$k = 1, 2, \ldots, \text{max\_iter}$}
    \State \textbf{// Store previous iteration values for coupling}
    \State $\redB{B^{k-1}} \gets \blueB{B^{k-1}}$, $\greenmu{\mu^{k-1}} \gets \greenmu{\mu^{k-1}}$, $P_B^{k-1} \gets P_B^{k-1}$
    \State
    \State \textbf{// Forward Pass: Solve each time step sequentially}
    \For{$t = 1, 2, \ldots, T$}
        \State Solve subproblem for time $t$ using:
        \State \hspace{2em} $\blueB{B^k[t-1]}$ from current forward pass (if $t > 1$)
        \State \hspace{2em} $\redB{B^{k-1}[t+1]}$, $\greenmu{\mu^{k-1}[t+1]}$ from previous iteration (if $t < T$)
        \State
        \State Extract solution:
        \State \hspace{2em} $\blueB{B^k[t]} \gets$ optimal SOC
        \State \hspace{2em} $P_B^k[t] \gets$ optimal battery power
        \State \hspace{2em} $P_{\text{subs}}^k[t] \gets$ optimal substation power
        \State \hspace{2em} $\greenmu{\mu^k[t]} \gets -\text{dual}(\text{battery\_dynamics}[t])$
        \State \hspace{2em} If $t = T$ with terminal constraint:
        \State \hspace{4em} $\greenmu{\mu^k[T]} \gets \greenmu{\mu^k[T]} + (-\text{dual}(\text{terminal\_soc}))$
        \State
        \If{$t < T$}
            \State $P_B^k[t+1] \gets$ optimal value of $P_B[t+1]$ variable
        \EndIf
    \EndFor
    \State
    \State \textbf{// Compute convergence metrics}
    \State $\text{err}_{\text{state}} \gets \|\blueB{B^k} - \blueB{B^{k-1}}\|_2$
    \State $\text{err}_{\text{dual}} \gets \|\greenmu{\mu^k} - \greenmu{\mu^{k-1}}\|_2$
    \State $\text{err}_{\text{KKT}} \gets \max(\text{err}_{\text{state}}, \text{err}_{\text{dual}})$
    \State
    \If{$\text{err}_{\text{KKT}} < \epsilon_{\text{tol}}$}
        \State \textbf{break} \Comment{Converged}
    \EndIf
\EndFor
\State
\State \textbf{Return:} $\{\blueB{B^k[t]}, P_B^k[t], P_{\text{subs}}^k[t], \greenmu{\mu^k[t]}\}_{t=1}^T$, convergence status
\end{algorithmic}
\end{algorithm}

\section{Convergence Criteria}

\subsection{State Trajectory Error}
\begin{align}
\text{err}_{\text{state}}^k = \left\| \blueB{B^k} - \blueB{B^{k-1}} \right\|_2 = \sqrt{\sum_{t=1}^T \left( \blueB{B^k[t]} - \blueB{B^{k-1}[t]} \right)^2}
\end{align}

\subsection{Dual Variable Error}
\begin{align}
\text{err}_{\text{dual}}^k = \left\| \greenmu{\mu^k} - \greenmu{\mu^{k-1}} \right\|_2 = \sqrt{\sum_{t=1}^T \left( \greenmu{\mu^k[t]} - \greenmu{\mu^{k-1}[t]} \right)^2}
\end{align}

\subsection{KKT Error (Combined)}
\begin{align}
\text{err}_{\text{KKT}}^k = \max\left( \text{err}_{\text{state}}^k, \text{err}_{\text{dual}}^k \right)
\end{align}

The algorithm converges when:
\begin{align}
\text{err}_{\text{KKT}}^k < \epsilon_{\text{tol}} \quad \text{(typically } \epsilon_{\text{tol}} = 10^{-5}\text{)}
\end{align}

\section{Objective Function}

\subsection{Total Objective (Physical Units)}
\begin{align}
J = \sum_{t=1}^T \left[ C[t] \cdot (P_{\text{subs}}[t] \cdot P_{\text{BASE}}) \cdot \Delta t + C_B \cdot (P_B[t] \cdot P_{\text{BASE}})^2 \cdot \Delta t \right]
\end{align}

\subsection{Objective Components}
\begin{align}
J_{\text{energy}} &= \sum_{t=1}^T C[t] \cdot (P_{\text{subs}}[t] \cdot P_{\text{BASE}}) \cdot \Delta t \quad \text{[\$]} \\
J_{\text{battery}} &= \sum_{t=1}^T C_B \cdot (P_B[t] \cdot P_{\text{BASE}})^2 \cdot \Delta t \quad \text{[\$]} \\
J &= J_{\text{energy}} + J_{\text{battery}} \quad \text{[\$]}
\end{align}

\subsection{Battery Quadratic Cost Rationale}
The battery cost term $C_B \cdot (P_B[t])^2$ serves multiple purposes:
\begin{enumerate}
    \item \textbf{Regularization}: Prevents numerical issues and excessive battery cycling
    \item \textbf{Smoothing}: Encourages gradual power transitions rather than bang-bang control
    \item \textbf{Conditioning}: Improves numerical stability of the optimization problem
    \item \textbf{Physical realism}: Approximates wear-and-tear and efficiency losses
\end{enumerate}

Typically set as: $C_B = 10^{-6} \times \min(C[t])$ to ensure it's a small regularization term.

\section{Dual Variable Extraction and Sign Conventions}

\subsection{Battery Dynamics Dual}
From the constraint:
\begin{align}
\blueB{B[t]} - B_{\text{prev}}[t] + \Delta t \cdot P_B[t] = 0
\end{align}

JuMP returns the dual $\lambda_{\text{dynamics}}[t]$. The shadow price is:
\begin{align}
\greenmu{\mu_{\text{dynamics}}[t]} = -\lambda_{\text{dynamics}}[t]
\end{align}

\textbf{Economic interpretation:} $\greenmu{\mu[t]} > 0$ means increasing $\blueB{B[t]}$ would \textit{decrease} the objective (energy stored is valuable). $\greenmu{\mu[t]} < 0$ means increasing $\blueB{B[t]}$ would \textit{increase} the objective (energy stored is costly).

\subsection{SOC Box Constraint Duals}
From constraints (negative inequality form):
\begin{align}
g_{\text{Bmin}}[t] &= \underline{B} - \blueB{B[t]} \leq 0 \quad \text{(i.e., } \blueB{B[t]} \geq \underline{B}\text{)} \\
g_{\text{Bmax}}[t] &= \blueB{B[t]} - \overline{B} \leq 0 \quad \text{(i.e., } \blueB{B[t]} \leq \overline{B}\text{)}
\end{align}

JuMP dual variables:
\begin{align}
\lambda_{\text{Bmin}}[t] &= -\text{dual}(g_{\text{Bmin}}[t]) \quad \text{(negate to get positive when active)} \\
\lambda_{\text{Bmax}}[t] &= \text{dual}(g_{\text{Bmax}}[t]) \quad \text{(already correct sign)}
\end{align}

Complementary slackness:
\begin{align}
\lambda_{\text{Bmin}}[t] \cdot (\blueB{B[t]} - \underline{B}) &= 0, \quad \lambda_{\text{Bmin}}[t] \geq 0 \\
\lambda_{\text{Bmax}}[t] \cdot (\overline{B} - \blueB{B[t]}) &= 0, \quad \lambda_{\text{Bmax}}[t] \geq 0
\end{align}

\subsection{Terminal SOC Constraint Dual}
If terminal constraint exists: $\blueB{B[T]} = B_{T,\text{target}}$

The total shadow price at $t = T$ is:
\begin{align}
\greenmu{\mu_{\text{total}}[T]} = \greenmu{\mu_{\text{dynamics}}[T]} + \greenmu{\mu_{\text{terminal}}[T]}
\end{align}
where:
\begin{align}
\greenmu{\mu_{\text{terminal}}[T]} = -\text{dual}(\text{terminal\_soc})
\end{align}

\section{KKT Optimality Conditions}

For each time step $t$, the KKT conditions are:

\subsection{Stationarity}
\begin{align}
\frac{\partial \mathcal{L}}{\partial P_{\text{subs}}[t]} &= C[t] \cdot \Delta t + \lambda_{\text{power\_balance}}[t] = 0 \\
\frac{\partial \mathcal{L}}{\partial P_B[t]} &= 2 C_B \cdot P_B[t] \cdot \Delta t + \lambda_{\text{power\_balance}}[t] + \Delta t \cdot \greenmu{\mu[t]} = 0 \\
\frac{\partial \mathcal{L}}{\partial \blueB{B[t]}} &= \greenmu{\mu[t]} - \greenmu{\mu[t+1]} + \lambda_{\text{Bmin}}[t] - \lambda_{\text{Bmax}}[t] = 0
\end{align}

For terminal step $t = T$ with terminal constraint:
\begin{align}
\frac{\partial \mathcal{L}}{\partial \blueB{B[T]}} &= \greenmu{\mu_{\text{dynamics}}[T]} + \greenmu{\mu_{\text{terminal}}[T]} + \lambda_{\text{Bmin}}[T] - \lambda_{\text{Bmax}}[T] = 0
\end{align}

\subsection{Primal Feasibility}
All constraints satisfied:
\begin{align}
P_{\text{subs}}[t] + P_B[t] - P_L[t] &= 0 \\
\blueB{B[t]} - B_{\text{prev}}[t] + \Delta t \cdot P_B[t] &= 0 \\
\underline{B} \leq \blueB{B[t]} &\leq \overline{B} \\
-P_{B,R} \leq P_B[t] &\leq P_{B,R}
\end{align}

\subsection{Dual Feasibility}
\begin{align}
\lambda_{\text{Bmin}}[t] &\geq 0 \\
\lambda_{\text{Bmax}}[t] &\geq 0
\end{align}

\subsection{Complementary Slackness}
\begin{align}
\lambda_{\text{Bmin}}[t] \cdot (\blueB{B[t]} - \underline{B}) &= 0 \\
\lambda_{\text{Bmax}}[t] \cdot (\overline{B} - \blueB{B[t]}) &= 0
\end{align}

\section{Comparison with Brute Force Solution}

\subsection{Centralized (Brute Force) Formulation}
The brute force approach solves the full multi-period problem in one shot:

\begin{align}
\min_{\{P_{\text{subs}}[t], P_B[t], B[t]\}_{t=1}^T} \quad & \sum_{t=1}^T \left[ C[t] \cdot P_{\text{subs}}[t] \cdot \Delta t + C_B \cdot (P_B[t])^2 \cdot \Delta t \right]
\end{align}

\textbf{Subject to:}
\begin{align}
P_{\text{subs}}[t] + P_B[t] &= P_L[t], \quad \forall t \in \mathcal{T} \\
B[1] - B_0 + \Delta t \cdot P_B[1] &= 0 \\
B[t] - B[t-1] + \Delta t \cdot P_B[t] &= 0, \quad \forall t \in \{2, \ldots, T\} \\
\underline{B} \leq B[t] &\leq \overline{B}, \quad \forall t \in \mathcal{T} \\
-P_{B,R} \leq P_B[t] &\leq P_{B,R}, \quad \forall t \in \mathcal{T} \\
B[T] &= B_{T,\text{target}} \quad \text{(if terminal constraint exists)}
\end{align}

\subsection{Key Differences}
\begin{itemize}
    \item \textbf{Brute Force}: Solves all time steps simultaneously, guarantees global optimum
    \item \textbf{DDP}: Solves time steps sequentially with iterative coupling, may require multiple passes to converge
    \item \textbf{Scalability}: DDP can parallelize across time steps (not implemented in current version)
    \item \textbf{Memory}: DDP has smaller subproblems, lower memory footprint per solve
    \item \textbf{Warm starting}: DDP can initialize with duals from brute force for rapid convergence
\end{itemize}

\section{Convergence Properties}

\subsection{Optimality Guarantee}
Under convexity assumptions (which hold for the copper plate problem):
\begin{itemize}
    \item If DDP converges (i.e., $\text{err}_{\text{KKT}} < \epsilon_{\text{tol}}$), the solution is \textbf{globally optimal}
    \item Convergence is guaranteed if the coupling penalty (implicitly through dual variables) is strong enough
    \item With warm start from brute force duals, DDP often converges in 1-2 iterations
\end{itemize}

\subsection{Convergence Rate}
\begin{itemize}
    \item \textbf{Linear convergence}: $\text{err}^{k+1} \leq \alpha \cdot \text{err}^k$ for some $\alpha < 1$
    \item Convergence rate depends on problem conditioning and coupling strength
    \item Tighter SOC bounds $\Rightarrow$ slower convergence (more binding constraints)
    \item More time periods $T$ $\Rightarrow$ longer horizon coupling, potentially slower convergence
\end{itemize}

\subsection{Practical Observations}
From numerical experiments with $T = 2, 4$ time periods:
\begin{itemize}
    \item \textbf{Cold start} ($\mu = 0$): Converges in 5-15 iterations to $\epsilon_{\text{tol}} = 10^{-5}$
    \item \textbf{Warm start} ($\mu$ from BF): Converges in 1-2 iterations
    \item \textbf{Objective accuracy}: Matches brute force to within $10^{-6}$ relative error
    \item \textbf{Solution accuracy}: State and control trajectories match BF to numerical precision
\end{itemize}

\section{Implementation Notes}

\subsection{Solver Choices}
\begin{itemize}
    \item \textbf{Gurobi}: Fast for quadratic programs (QP), commercial license required
    \item \textbf{Ipopt}: Open-source nonlinear solver, slightly slower but freely available
    \item Both solvers produce identical solutions for this convex problem
\end{itemize}

\subsection{Numerical Considerations}
\begin{itemize}
    \item Use per-unit system to avoid ill-conditioning from large physical values
    \item Set $C_B$ small enough to be a regularization, not a dominant cost
    \item Terminal SOC constraint should not be at SOC bounds to avoid numerical issues
    \item Monitor dual variable magnitudes to detect potential infeasibility
\end{itemize}

\subsection{Parallelization Potential}
Current implementation is \textit{sequential} forward pass. Future extensions:
\begin{itemize}
    \item Parallelize subproblems within each iteration (requires asynchronous updates)
    \item Use GPU acceleration for large-scale systems
    \item Implement backward pass for improved convergence
\end{itemize}

\section{Example Problem Instance}

\subsection{System Configuration}
\begin{align}
T &= 4 \text{ time periods} \\
\Delta t &= 1.0 \text{ hour} \\
P_{\text{BASE}} &= 1000 \text{ kW} \\
E_{\text{BASE}} &= 1000 \text{ kWh}
\end{align}

\subsection{Battery Specifications}
\begin{align}
E_{\text{Rated}} &= 4000 \text{ kWh} = 4.0 \text{ pu} \\
P_{B,R} &= 800 \text{ kW} = 0.8 \text{ pu} \\
\text{SOC}_{\min} &= 0.30 \quad (30\%) \\
\text{SOC}_{\max} &= 0.95 \quad (95\%) \\
B_0 &= 0.30 \times 4.0 = 1.2 \text{ pu} \quad (30\% \text{ initial SOC}) \\
B_{T,\text{target}} &= 0.32 \times 4.0 = 1.28 \text{ pu} \quad (32\% \text{ terminal SOC})
\end{align}

\subsection{Load Profile (Rated = 1250 kW)}
\begin{align}
P_L[t] &= 1250 \times \text{LoadShape}[t] / 1000 \quad \text{[pu]} \\
\text{LoadShape} &= 0.8 + 0.2 \times \frac{\sin(\omega t - 0.8) + 1}{2}, \quad \omega = \frac{2\pi}{T}
\end{align}

Example for $T = 4$: $P_L \approx [1.00, 1.10, 1.15, 1.05]$ pu (approximate sinusoidal)

\subsection{Price Profile (Charge-Discharge Scenario)}
Design to test look-ahead capability:
\begin{align}
C[t] = \begin{cases}
0.05 \text{ \$/kWh} & t \in \{1, 2\} \quad \text{(cheap: charge)} \\
0.20 \text{ \$/kWh} & t \in \{3, 4\} \quad \text{(expensive: discharge)}
\end{cases}
\end{align}

This creates a 4× price difference to incentivize:
\begin{itemize}
    \item \textbf{Periods 1-2}: Charge battery (low prices)
    \item \textbf{Periods 3-4}: Discharge battery (high prices)
\end{itemize}

\subsection{Battery Cost Coefficient}
\begin{align}
C_B = 10^{-6} \times \min(C[t]) = 10^{-6} \times 0.05 = 5 \times 10^{-8} \text{ \$/kW}^2\text{/h}
\end{align}

\section{Appendix: Physical Interpretation}

\subsection{Battery Power Sign Convention}
\begin{itemize}
    \item $P_B[t] > 0$: Battery \textbf{discharging} (injecting power into grid)
    \item $P_B[t] < 0$: Battery \textbf{charging} (consuming power from grid)
    \item $P_B[t] = 0$: Battery idle
\end{itemize}

\subsection{Economic Signals}
\begin{itemize}
    \item High $C[t]$ $\Rightarrow$ Expensive energy $\Rightarrow$ Discharge battery ($P_B[t] > 0$) $\Rightarrow$ Reduce $P_{\text{subs}}[t]$
    \item Low $C[t]$ $\Rightarrow$ Cheap energy $\Rightarrow$ Charge battery ($P_B[t] < 0$) $\Rightarrow$ Increase $P_{\text{subs}}[t]$
    \item $\greenmu{\mu[t]} > 0$ $\Rightarrow$ Energy stored at $t$ is valuable (future high prices)
    \item $\greenmu{\mu[t]} < 0$ $\Rightarrow$ Energy stored at $t$ is costly (future low prices)
\end{itemize}

\subsection{SOC Trajectory Interpretation}
\begin{itemize}
    \item $\blueB{B[t]}$ decreases when $P_B[t] > 0$ (discharging)
    \item $\blueB{B[t]}$ increases when $P_B[t] < 0$ (charging)
    \item Optimal trajectory balances energy arbitrage profit against SOC constraints
    \item Terminal constraint $B[T] = B_{T,\text{target}}$ ensures battery returns to desired state
\end{itemize}

\end{document}
